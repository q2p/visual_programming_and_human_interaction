\documentclass[12p]{article}
\usepackage[utf8x]{inputenc}
\usepackage[T2A]{fontenc}
\usepackage[russian]{babel}
\usepackage[a4paper, margin=20mm]{geometry}
\usepackage{minted}

\setlength{\parindent}{0em}
\setlength{\parskip}{1em}

\begin{document}

\title{Стиль кода}
\author{Автор: Озеров Данил Алексеевич, ИА-032,\\ email: danilozerovmail@gmail.com,\\ github: @q2p}
\date{Февраль 2022}

\maketitle

\section{Введение}

При разработке личных проектов я стараюсь использовать одинаковый стиль кода вне зависимости от языка.
Я решил привести примеры на различных языках, так как каждый из них имеет свои ньюансы при применении различных стилистических приёмов.

\section{Табуляция}

Для отступов внутри блоков кода используются табы. Размер табов в редакторе - два пробела.\newline

\textbf{Rust:}
\begin{minted}{rust}
fn main() {
  println!("Hello")
}
\end{minted}

Если же отступы нужны внутри строки, то внутри строк используются пробелы.\newline

\textbf{Rust:}
\begin{minted}{rust}
const fn rotate_y_matrix(angle: f32) -> &'static [f32; 9] {
  return &[
     angle.cos(), 0, angle.sin(),
               0, 1,           0,
    -angle.sin(), 0, angle.cos(),
  ];
}
\end{minted}
\pagebreak

\section{Именование}

Переменные, функции, модули, файлы именуются используя \mintinline{c}{snake_case}. Константы именуются используя \mintinline{c}{CONSTANT_CASE}.\newline

\textbf{C:}
\begin{minted}{c}
enum {
  MEANING_OF_LIFE = 42,
};

uint32_t get_random_number() {
  return 4; // chosen by fair dice roll.
            // guaranteed to be random.
}
\end{minted}

\section{Пробелы рядом со специальныи символами}

Знаки арифметических операций обрамляются пробелами с обеих сторон.\newline

\textbf{Rust:}
\begin{minted}{rust}
let r = (a + b) * c;
\end{minted}

Скобки при вызове функций не обрамлсяются пробелом слева. Операции взятия указателя ставятся рядом с переменной. После запятой ставится пробел.\newline

\textbf{Rust:}
\begin{minted}{rust}
call_func(a, &b);
\end{minted}

C-подобные языки позволяют убрать фигурные скобки при использовании блока кода, если блок состоит только из одной строки. Я скобки оставляю.\newline

\textbf{Rust:}
\begin{minted}{rust}
for i in list {
  i.do_stuff();
}
\end{minted}

Скобки открывающие блок не переносятся на новую строку.\newline

\textbf{Rust:}
\begin{minted}{rust}
if a {
  println!("а")
} else {
  println!("неа")
}
\end{minted}
\pagebreak

При объявлении указателя, звёздочка ставится рядом с типом.\newline

\textbf{C:}
\begin{minted}{c}
void func(uint32_t* pointer);
\end{minted}

\section{Лямбда выражения}

Если в языке есть нативная конструкция, то лучше использовать её вместо функций с лямбдами:\newline

\textbf{Kotlin:}
\begin{minted}{kotlin}
for(i in list) {
  println(i)
}
\end{minted}

Вместо:\newline

\textbf{Kotlin:}
\begin{minted}{kotlin}
list.forEach {
  println(it)
}
\end{minted}

\section{Инициализация переменных}

Переменные объявляются только тогда когда становятся необходимыми и живут только в том блоке, в котором они нужны.\newline

\textbf{Rust:}
\begin{minted}{rust}
fn do_the(funny: u64) {
  for i in 0..funny {
    let temp = (i as f32).sqrt();
    println!("{}", temp);
  }
}
\end{minted}

Вместо:\newline

\textbf{Rust:}
\begin{minted}{rust}
fn do_the(funny: u64) {
  let mut temp;
  for i in 0..funny {
    temp = (i as f32).sqrt();
    println!("{}", temp);
  }
}
\end{minted}

\section{Неизменяемые переменные}

Если язык позволяет, то неизменяемые переменные помечаются как таковые.\newline

\textbf{TypeScript:}
\begin{minted}{typescript}
function power_of_4(a: number): number {
  const power_of_2 = a * a;
  return power_of_2 * power_of_2;
}
\end{minted}

\section{Константы в C и C++}

В C и C++ вместо макросов или переменных с модификатором \mintinline{c}{const} используются энумераторы, так как они вычисляются во время компиляции, их нельзя изменить вызвав UB и они не подвержены недостаткам макросов.\newline

\textbf{C:}
\begin{minted}{c}
enum {
  MATH_PI         = 3.1415927,
  MATH_E          = 2.7182818,
  MATH_PI_TIMES_E = MATH_PI * MATH_E,
};
\end{minted}

\section{Неоднозначный размер типов}

Если язык по умолчанию не имеет жёстких ограничений по размеру типов \cite{data_type_sizes}, но такие ограничения можно включить, то используются типы с однозначными размерами.\newline

\textbf{C:}
\begin{minted}{c}
// Может иметь 32, 36 или 48 бит точности
// в зависимости от архитектуры и компилятора
long a;
// Всегда имеет 32 бита точности
int32_t b;
\end{minted}

\section{\#ifndef guard}

Для избежания циклических включений заголовочных файлов в C и C++ используется не стандартный \mintinline{c}{#pragma once} \newline

\textbf{C:}
\begin{minted}{c}
#pragma once

void my_function(in8_t a);
\end{minted}

\section{Вермя жизни}

Если есть возможность использовать только чистые функции или синглтон, то лучше этим воспользоваться.\newline

\textbf{Rust:}
\begin{minted}{rust}
fn make_request(db_connection: DBConnection) -> Option<String> {
  return db_connection.make_request().as_ref();
}
\end{minted}

Вместо того, чтобы использовать излишние билдеры или фабрики.\newline

\textbf{Rust:}
\begin{minted}{rust}
struct DBRequest {
  db_connection: DBConnection,
  result: Option<String>,
}
impl DBRequest {
  fn new(db_connection: DBConnection) -> DBRequest {
    return DBRequest {
      db_connection: db_connection,
      result: None,
    };
  }
  fn make_request(&mut self) {
    self.result = self.db_connection.make_request();
  }
  fn get_response(&self) -> Option<&str> {
    return self.result.as_ref();
  }
}
\end{minted}

\begin{thebibliography}{}

\bibitem{c_const_override}
ISO/IEC 9899:2018 - C, \S6.7.3 [Электронный ресурс] URL: http://www.open-std.org/jtc1/sc22/wg14/www/docs/n2310.pdf (дата обращения: 06.02.2022).

\bibitem{data_type_sizes}
Joe Nelson - C Portability Lessons from Weird Machines [Электронный ресурс] URL: https://begriffs.com/posts/2018-11-15-c-portability.html (дата обращения: 06.02.2022).

\bibitem{kotlin_stdlib}
Kotlin Standard Library Documentation [Электронный ресурс] URL: https://kotlinlang.org/api/latest/jvm/stdlib (дата обращения: 06.02.2022).

\bibitem{minted}
G. M. Poore The minted package: Highlighted source code in LaTeX [Электронный ресурс] URL: http://tug.ctan.org/tex-archive/macros/latex/contrib/minted/minted.pdf (дата обращения: 06.02.2022).

\end{thebibliography}

\end{document}
